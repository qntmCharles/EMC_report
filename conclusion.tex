\chapter{Conclusion}
\section{Comments}
To conclude: there is yet a conclusion to be made. Throughout the 6 studies I present, small sample sizes have been an issue. Better global coverage, data over longer periods, and more data in general, would provide a much better analysis and could confirm or refute results I have found. I would like to note that this entire report is an analysis: I do not put forward explanations for all of my findings. Further modelling, analysis and investigation is a necessity. However, saying this, there are interesting findings in this report, supporting arguments for previous studies, and curious results that beg further study.
\section{Main findings}
\subsection{Diurnal shift}
\begin{itemize}
	\item I have presented a model, which fits the data. I model diurnal shift as a result of orbital velocity variations impacting incident meteor velocities.
	\item The peak hour of diurnal shift is at 6am local time for an observer.
	\item The intensity of diurnal shift appears to be uniform across the globe, with no correlation to longitude or latitude.
	\item There is no clear link between the amplitude of diurnal shift and latitude, though some locations have a clearly larger amplitude than others.
\end{itemize}
\subsection{Spacial variation of data characteristics}
\begin{itemize}
	\item Data characteristics are not well correlated with latitude or longitude.
	\item Overall, there is little variation in data characteristics, suggesting a uniform quality of observation across the globe.
	\item The distribution of meteor counts is generally symmetric.
\end{itemize}
\subsection{Temporal variation of radar counts \& data characteristics}
\begin{itemize}
	\item There is a significant increase of around 2$\times$ of hourly meteor detection counts between 2005 and 2011.
	\item The maximum in counts correlates well with a minimum in the solar cycle.
	\item Daytime and night-time differences in detection counts are not significant (1\% significance level).
	\item Meteor counts increase towards the middle of the year for all observers, though this effect is most likely due to observers from the North Hemisphere, suggesting it is the summer months.
	\item There is little variation in meteor counts over a month.
\end{itemize}
\subsection{Antenna data characteristics}
\begin{itemize}
	\item There is no clear correlation between data characteristics and antenna type.
	\item Some antennas appear to have more extreme values than others, though not significantly.
	\item Generally, the antenna type appears to have little impact on the quality of radio meteor detection.
\end{itemize}
\subsection{Zenithal Hourly Rate (ZHR) validity}
\begin{itemize}
	\item Using the modifications I propose, the formula for normalising an observer's hourly rate to ZHR is valid.
	\item The modified formula produces radio ZHRs that correlate well with visual ZHRs.
	\item I present an alternative function for radiant height correction, accompanied with a geometrical argument.
	\item Further work is required on the validity of models of meteor shower population index.
\end{itemize}
\subsection{Meteor detection by RMS difference image analysis}
\begin{itemize}
	\item RMS difference image analysis provides a valid comparative measure of meteor flux, which correlates well with other methods of radio meteor detection.
	\item RMS difference provides a suitable alternative to software that counts detected meteors.
	\item Alternative methods may provide an improvement, such as edge detection, or structural similarity index measures.
\end{itemize}

\section{Acknowledgements}
I am indebted to the RMOB organisation \cite{rmob}, from which I have sourced my data, and been given permission to complete this study. I am also grateful for the assistance of Malcolm Simpson and Ed Horncastle in the process of the studies I present.